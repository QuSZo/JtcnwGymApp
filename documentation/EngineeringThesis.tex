\documentclass{article}

% Language setting
% Replace `english' with e.g. `spanish' to change the document language
\usepackage{polski}
\usepackage[utf8]{inputenc}

% Set page size and margins
% Replace `letterpaper' with `a4paper' for UK/EU standard size
\usepackage[letterpaper,top=2cm,bottom=2cm,left=3cm,right=3cm,marginparwidth=1.75cm]{geometry}

% Useful packages
\usepackage{amsmath}
\usepackage{graphicx}
\usepackage[colorlinks=true, allcolors=blue]{hyperref}

\title{Praca inżynierska}
\author{Mikołaj Kuszowski}

\begin{document}
\maketitle

\section{Spotkanie z biznesem}
      Pierwszym etapem stworzenia aplikacji jest konfrontacja IT z Biznesem. W naszym wypadku Biznesem mogą być klienci czyli zewnętrzne firmy lub inne działy wewnątrz firmy. Biznes przychodzi z problemem, przedstawia swoje cele i wymagania. Dział IT musi więc dobrze rozumieć cele i potrzeby biznesowe, natomiast biznes musi wiedzieć, czy organizacja dysponuje zasobami niezbędnymi do ich realizacji.[https://cyfrowa.rp.pl/technologie/art40278241-konieczna-wspolpraca-miedzy-it-i-biznesem] Łącznikiem między Binzesem, a IT jest analityk biznesowy, który ma wiedzę techniczną oraz jest w stanie w przejrzysty sposób zrozumieć wymagania biznesu.

      Proces wytwarzania oprogramowania możemy rozpocząć od warsztatu o fachowej nazwie "Event Storming". Podczas takiego spotkania uczestniczą:
      \begin{itemize}
            \item Eksperci domenowi - od strony biznesu
            \item Eksperci techniczni - np. analitycy biznesowi
      \end{itemize}

      Sama forma warsztatu wydaje się dość prosta – do dyspozycji mamy dużą tablicę czy ścianę oraz mnóstwo różnokolorowych karteczek. Każdy z uczestników identyfikuje zdarzenia („Domain Events”), które występują w trakcie działania programu – np. „Utworzono konto użytkownika”, „Zamówiono towar”, czy „Wygenerowano fakturę”. Zdarzenia te zapisuje się na karteczkach i umieszcza na tablicy. [https://bulldogjob.pl/readme/event-storming-pierwszy-krok-do-ddd]. Celem takiego spotkania jest klarowne poznanie domeny (biznesu) i zrozumienie potrzeb oraz wymagań.

      Wytwarzanie aplikacji zaczniemy od procesu Event Storimngu. Aby ją lepiej zrozumień zdefiniujmy co oznaczają poszczególne kolory karteczek.
      \begin{itemize}
            \item Żółty:
            \item itd.
      \end{itemize}
      



\end{document}